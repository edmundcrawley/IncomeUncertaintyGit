\newcommand{\econtexRoot}{Paper/}
% The \commands below are required to allow sharing of the same base code via Github between TeXLive on a local machine and ShareLaTeX.  This is an ugly solution to the requirement that custom LaTeX packages be accessible, and that ShareLaTeX seems to ignore symbolic links (even if they are relative links to valid locations)
\providecommand{\econtex}{\econtexRoot/texmf-local/tex/latex/econtex}
\providecommand{\econtexSetup}{\econtexRoot/texmf-local/tex/latex/econtexSetup}
\providecommand{\econtexShortcuts}{\econtexRoot/texmf-local/tex/latex/econtexShortcuts}
\providecommand{\econtexBibMake}{\econtexRoot/texmf-local/tex/latex/econtexBibMake}
\providecommand{\econtexBibStyle}{\econtexRoot/texmf-local/bibtex/bst/econtex}
\providecommand{\notes}{\econtexRoot/texmf-local/tex/latex/handout}
\providecommand{\handoutSetup}{\econtexRoot/texmf-local/tex/latex/handoutSetup}
\providecommand{\handoutShortcuts}{\econtexRoot/texmf-local/tex/latex/handoutShortcuts}
\providecommand{\handoutBibMake}{\econtexRoot/texmf-local/tex/latex/handoutBibMake}
\providecommand{\handoutBibStyle}{\econtexRoot/texmf-local/bibtex/bst/handout}

  

\documentclass[titlepage,abstract]{econtex}
\usepackage{econtexSetup}
\usepackage{econtexShortcuts}

\begin{document}\bibliographystyle{\econtexBibStyle}
	
\section{MPC}
\cite{gelman_what_2016}
Uses app data. Transitory MPC out of tax refunds. MPC circa 15\% over 3 months. Strong relationship between cash on hand and MPC.\\
\\
\cite{koustas_consumption_2018}
Uses rideshare data. Finds responsiveness to income of 30\%, much reduced by smoothing via secondary job.\\
\\
\cite{gelman_response_2016}
Estimates permanent MPC from gasoline prices. Estimates MPC of 1.\\
\\
\cite{gelman_how_2015}
Used temporary shut down of govt. to identify spending response. 50\% reduction!! But lots of this is thought to be accessing short term liquidity. Not paying mortgage particularly relevant...\\
\\
\cite{baker_debt_2015}
Find MPC to semi-permanent shocks of 40\% \\
\\
\cite{ganong_consumer_2017}
Looks at consumption behavior around unemployment benefits (strong response to predictable benefits) \\
\\
\cite{fuster_what_2018}
Asks people in NY Fed Survey of Consumer Expectations what they would do with \$500. 8\% for a gain of \$500, 30\% for a loss of \$500.



\section{BPP Methodology}
\cite{blundell_consumption_2008} \\
\href{url}{http://www.ucl.ac.uk/~uctp39a/BBP\%20AER.pdf} \\
This is the original BPP paper. The online appendix is also useful. \\
\href{url}{https://assets.aeaweb.org/assets/production/articles-attachments/aer/data/dec08/20050545\_app.pdf} \\
\\
This set of slides is useful for understanding BPP and other methods in a similar vein: \\
\href{url}{https://ocw.mit.edu/courses/economics/14-772-development-economics-macroeconomics-spring-2013/lecture-videos-and-slides/MIT14\_772S13\_lecture8.pdf} \\
\\
\cite{kaplan_how_2010} \\
\href{url}{https://www.aeaweb.org/articles?id=10.1257/mac.2.4.53} \\
Paper by Kaplan and Violante examining how accurate the BPP methodology is in the standard incomplete market model. Also useful for understanding how BPP works and the assumptions made.\\
\\
\cite{violante_wealthy_2014} \\
\href{url}{http://www.nber.org/papers/w20073} \\
Wealty-hand-to-mouth paper uses the BPP methodology in section 7 to compare MPC for agents of different hand-to-mouth status. \\
\\
\cite{attanasio_is_1995} \\
\href{url}{http://www.nber.org/papers/w4795} \\
Earlier paper about whether or not people have insurance against shocks to their groups' income.\\
\\
How Much Consumption Insurance in the U.S \\
\href{url}{http://www.artsrn.ualberta.ca/econweb/hryshko/Papers/HM\_2017\_06\_19.pdf} \\
Paper on BPP methodology using PSID data - criticizing the assumption of permanent shocks in non-sample households. Finds significant AR(1) characteristics that bias results of permanent shock down a lot. \\
\\
\cite{heathcote_unequal_2010} \\
\href{url}{http://www.sciencedirect.com/science/article/pii/S1094202509000659} \\
Focus on income dynamics - compares BPP method with others used. In particular shows the `differences' method of BPP gives very different (and implausibly large) estimates of permanent shock variance.\\
\\
\cite{carroll_nature_1997} \\
\href{url}{http://www.econ2.jhu.edu/people/ccarroll/nature.pdf} \\
Decomposing income into permanent and transitory shocks \\
\\
\cite{commault_how_2017}
\href{url}{https://www.dropbox.com/s/3m0wsuygu21mq3e/Commault\%20-\%20JMP.pdf?dl=0} \\
Paper trying to explain the difference between literature and BPP\\
\\
\cite{moffitt_trends_2012} \\
href{url}{http://www.econ2.jhu.edu/people/Moffitt/mg2\_0795.pdf} \\
Moffitt early paper decompsing income - trends in covariance structure\\
\\
\cite{working_note_1960} \\
\href{url}{http://www.e-m-h.org/Work60.pdf} \\
Working's 1960 paper on time aggregation\\
\\
\cite{meghir_income_2004} \\
\href{url}{https://pdfs.semanticscholar.org/feea/84af8be128de04f4279eb3a77084c8e8e9bc.pdf} \\
Maghir and Pistaferri "Income variance Dynamics and Heterogeneity" breaks down income process\\
\\
\cite{nielsen_impact_2004} \\
\href{url}{http://faculty.haas.berkeley.edu/vissing/nielsen\_vissing2006.pdf} \\
The Impact of Labor Income Risk on Educational Choices: Estimates and Implied Risk Aversion. Paper from Denmark that estimates permanent and transitory risk variance using the Danish data\\
\\
\cite{heathcote_unequal_2010} \\
\href{url}{http://www.nber.org/papers/w15483.pdf} \\
Heathcote and Perri and Violante overview of inequality in the US, describing income decomposition\\
\\
\cite{noauthor_imputing_nodate} \\
\href{url}{https://web.stanford.edu/~pista/impute.pdf} \\
Blundell and Pistaferri using CEX to impute consumption from food data in PSID\\
\\
\cite{arellano_earnings_2017} \\
\href{url}{http://www.ucl.ac.uk/~uctp39a/ABB\_Ecta\_May\_2017.pdf} \\
New paper by Arellano, Bludell and Bonhomme using quantile regressions to estimate persistent income process. Not clear how to generalize to deal with time aggregation problem \\
\\
\cite{storesletten_consumption_2004} \\
\href{url}{http://www.nber.org/papers/w7995.pdf} \\
Storesletten: consumption and risk sharing over the lifecycle. A little like an earlier version of BPP - looks at consumption inequality over the lifecycle in PSID data \\
\\
\cite{vidangos_rising_2013} \\
\href{url}{https://www.brookings.edu/wp-content/uploads/2016/07/2013a\_panousi.pdf} \\
Rising Inequality: Transitory or Persistent? New Evidence from a Panel of U.S. Tax Returns. Great data \\
\\
\cite{moffitt_income_2018} \\
Moffit's survey paper on income volatility in the PSID data

\section{MPC Heterogeneity and Relation to Macro}
\cite{auclert_monetary_2015} \\
\href{url}{https://web.stanford.edu/~aauclert/mp\_redistribution.pdf} \\
Lays down theory as to how the distribution of the MPC changes the transmission mechanism of monetary policy. In particular the covariance of MPC with unhedged interest rate exposure is on of the `sufficient statistics' needed. It is possible we can measure this.\\
\\
\cite{kaplan_monetary_2016} \\
\href{url}{https://www.ecb.europa.eu/pub/pdf/scpwps/ecbwp1899.en.pdf} \\
Full two-sector Heterogenous New Keynesian Model (HANK) that shows the monetary policy transmission mechanism is driven primarily by heterogeneity in the MPC.\\
\\
\cite{carroll_distribution_2016} \\
\href{url}{http://www.econ2.jhu.edu/people/ccarroll/papers/cstwMPC/} \\
Calibrates a standard incomplete markets model to US wealth data using heterogenous discount rates. Shows you can get a large mean MPC even with high levels of aggregate wealth.\\
\\
\cite{wong_population_2016}\\
\href{url}{https://www.dropbox.com/s/nwu6d2d00cmzh99/Arlene\_Wong\_Aging\_Latest.pdf?dl=0} \\
Recent successful job market paper about heterogenous responses to monetary policy, particularly involving different MPC around house purchasing.\\
\\
\section{Measuring MPC}
\cite{jappelli_consumption_2010} \\
\href{url}{http://www.nber.org/papers/w15739} \\
Provides a summary of attempts to measure MPC using different methods.\\
\\
\cite{broda_economic_2014} \\
\href{url}{http://www.nber.org/papers/w20122} \\
Effect of 2008 stimulus payments using Nielson Consumer data.\\
\\
\cite{fagereng_mpc_2016}\\
\href{url}{https://www.ssb.no/en/forskning/discussion-papers/\_attachment/286054?\_ts=158af859c98}
Andreas' paper on MPC out of lottery payments according to balance sheet characteristics.\\
\\
\cite{parker_consumer_2013}\\
\href{url}{http://www.nber.org/papers/w16684.pdf}
Parker et al on Stimulus checks of 2008.\\
\\
\cite{agarwal_consumption_2014}\\
\href{url}{http://ushakrisna.com/AER\_Sing.pdf}
Singapore growth dividend checks, finds 90\% MPC over 10 months.\\
\\
\cite{lupton_household_2005}\\
\href{url}{https://www.federalreserve.gov/pubs/FEDS/2005/200532/200532pap.pdf}
The Household Spending Response to the 2003 Tax Cut: Evidence from Survey Data.\\
\\


\section{Imputing Consumption Data}
\cite{browning_imputing_2003} \\
\href{url}{http://onlinelibrary.wiley.com/doi/10.1111/1468-0297.00135/abstract} \\
Paper about the imputation method used on Danish data.\\
\\
\cite{fagereng_imputing_2015} \\
\href{url}{https://www.ssb.no/en/forskning/discussion-papers/\_attachment/249164?\_ts=15185bf7d90} \\
Andreas' paper about his imputation method \\
\\
What Can We Learn About Household Consumption From Information on Income and Wealth \\
\href{url}{https://sites.google.com/site/magnemogstad/Measuring\_Consumption\_final.pdf?attredirects=0\&d=1}
Magne's paper with different method of computing consumption from Norwegian data\\
\\
\section{Consumption Smoothing}
\cite{campbell_why_1989} \\
\href{url}{https://www.jstor.org/stable/2297552?seq=1\#page\_scan\_tab\_contents} \\
Paper points out that given the persistence in aggregate income growth, aggregate consumption should be less smooth than income. At a micro level this is not the case. Not sure how relevant this is.\\
\\
\cite{campbell_consumption_1989} \\
\href{url}{http://www.nber.org/chapters/c10965.pdf} \\
Points out the `excess sensitivity' to expected income changes. Very basic two-agent model where one is hand-to-mouth and the other is fully rational.\\
\\
\cite{carroll_precautionary_2009} \\
\href{url}{http://www.nber.org/papers/w8233} \\
Calculates the MPC out of permanent shocks in the standard incomplete markets model to be between 0.85 and 0.95 (compare with 0.67 from BPP) \\






\bibliography{AllPapers}
%\input econtexBibMake

\end{document}